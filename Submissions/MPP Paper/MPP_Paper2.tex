\documentclass[english,man]{apa6}

\usepackage{amssymb,amsmath}
\usepackage{ifxetex,ifluatex}
\usepackage{fixltx2e} % provides \textsubscript
\ifnum 0\ifxetex 1\fi\ifluatex 1\fi=0 % if pdftex
  \usepackage[T1]{fontenc}
  \usepackage[utf8]{inputenc}
\else % if luatex or xelatex
  \ifxetex
    \usepackage{mathspec}
    \usepackage{xltxtra,xunicode}
  \else
    \usepackage{fontspec}
  \fi
  \defaultfontfeatures{Mapping=tex-text,Scale=MatchLowercase}
  \newcommand{\euro}{€}
\fi
% use upquote if available, for straight quotes in verbatim environments
\IfFileExists{upquote.sty}{\usepackage{upquote}}{}
% use microtype if available
\IfFileExists{microtype.sty}{\usepackage{microtype}}{}

% Table formatting
\usepackage{longtable, booktabs}
\usepackage{lscape}
% \usepackage[counterclockwise]{rotating}   % Landscape page setup for large tables
\usepackage{multirow}		% Table styling
\usepackage{tabularx}		% Control Column width
\usepackage[flushleft]{threeparttable}	% Allows for three part tables with a specified notes section
\usepackage{threeparttablex}            % Lets threeparttable work with longtable

% Create new environments so endfloat can handle them
% \newenvironment{ltable}
%   {\begin{landscape}\begin{center}\begin{threeparttable}}
%   {\end{threeparttable}\end{center}\end{landscape}}

\newenvironment{lltable}
  {\begin{landscape}\begin{center}\begin{ThreePartTable}}
  {\end{ThreePartTable}\end{center}\end{landscape}}

  \usepackage{ifthen} % Only add declarations when endfloat package is loaded
  \ifthenelse{\equal{\string man}{\string man}}{%
   \DeclareDelayedFloatFlavor{ThreePartTable}{table} % Make endfloat play with longtable
   % \DeclareDelayedFloatFlavor{ltable}{table} % Make endfloat play with lscape
   \DeclareDelayedFloatFlavor{lltable}{table} % Make endfloat play with lscape & longtable
  }{}%



% The following enables adjusting longtable caption width to table width
% Solution found at http://golatex.de/longtable-mit-caption-so-breit-wie-die-tabelle-t15767.html
\makeatletter
\newcommand\LastLTentrywidth{1em}
\newlength\longtablewidth
\setlength{\longtablewidth}{1in}
\newcommand\getlongtablewidth{%
 \begingroup
  \ifcsname LT@\roman{LT@tables}\endcsname
  \global\longtablewidth=0pt
  \renewcommand\LT@entry[2]{\global\advance\longtablewidth by ##2\relax\gdef\LastLTentrywidth{##2}}%
  \@nameuse{LT@\roman{LT@tables}}%
  \fi
\endgroup}


\ifxetex
  \usepackage[setpagesize=false, % page size defined by xetex
              unicode=false, % unicode breaks when used with xetex
              xetex]{hyperref}
\else
  \usepackage[unicode=true]{hyperref}
\fi
\hypersetup{breaklinks=true,
            pdfauthor={},
            pdftitle={The generalization of abstract verb meaning: Adults and 4-5 year old children show plasticity in verb biases that extend across semantic fields},
            colorlinks=true,
            citecolor=blue,
            urlcolor=blue,
            linkcolor=black,
            pdfborder={0 0 0}}
\urlstyle{same}  % don't use monospace font for urls

\setlength{\parindent}{0pt}
%\setlength{\parskip}{0pt plus 0pt minus 0pt}

\setlength{\emergencystretch}{3em}  % prevent overfull lines

\ifxetex
  \usepackage{polyglossia}
  \setmainlanguage{}
\else
  \usepackage[english]{babel}
\fi

% Manuscript styling
\captionsetup{font=singlespacing,justification=justified}
\usepackage{csquotes}
\usepackage{upgreek}

 % Line numbering
  \usepackage{lineno}
  \linenumbers


\usepackage{tikz} % Variable definition to generate author note

% fix for \tightlist problem in pandoc 1.14
\providecommand{\tightlist}{%
  \setlength{\itemsep}{0pt}\setlength{\parskip}{0pt}}

% Essential manuscript parts
  \title{The generalization of abstract verb meaning: Adults and 4-5 year old
children show plasticity in verb biases that extend across semantic
fields}

  \shorttitle{Generalization of verb meaning}


  \author{Melissa Kline\textsuperscript{1,2}, Amy Geojo, Annelot de Rechteren van Hemert\textsuperscript{3}, \& Jesse Snedeker\textsuperscript{2}}

  \def\affdep{{"", "", "", ""}}%
  \def\affcity{{"", "", "", ""}}%

  \affiliation{
    \vspace{0.5cm}
          \textsuperscript{1} Massachusetts Institute of Technology\\
          \textsuperscript{2} Harvard University\\
          \textsuperscript{3} TODO: Annelot's current institution  }

  \authornote{
    \newcounter{author}
    These would be my acknowledgements when the paper was finished.

                      Correspondence concerning this article should be addressed to Melissa Kline, 3037D 77 Massachusetts Ave., Cambridge, MA, 02139. E-mail: \href{mailto:mekline@mit.edu}{\nolinkurl{mekline@mit.edu}}
                                              }


  \abstract{Enter abstract here. Each new line herein must be indented, like this
line.

How do we break down representations of events to encode them in
language? Across languages, most verbs encode either Ends (e.g.~what
happens, crossing the floor) or Means (e.g.~how it happens, by dancing)
of an event, but not both (cf.~Talmy, 1985). Havasi et al. (2014) showed
these biases are not fixed but malleable -- when adults and 4-6yos learn
several verbs in a row with path meanings (rise, cross), they begin to
guess subsequent novel verbs will refer to path as well. For adults,
these biases are very abstract: after adults learned a path bias for
motion events, they preferred Ends verbs for change-of-state scenes as
well (Geojo 2015). Accomplishing this requires some kind of very general
representation of events that can account for hitting (manner-of-action)
being more like running (manner-of-motion) than like entering (path).

Pre-linguistic infants are sensitive to a non-linguistic means/ends
distinction (Phillips \& Wellman, 2005; Woodward, 1998, Gergely et al.
2002), but we do not know whether this early conceptual framework
provides a foundation for learning verb semantics. Are parallels between
means/end structure across domains a late-learned cognitive skill, or do
they emerge early in development? 4-6-yo children (N=58) were presented
with a repeating learning sequence (Figure 1):

Bias/new verb test: A word/event pairing is presented (e.g.~comb-rip,
gorping); children choose whether gorping means an event maintaining
either action (comb-flatten) or effect (hammer-rip).

Training: 3 additional events provide evidence for one interpretation
(e.g.~effect, rip)

Same-verb Test: 2 new events matching either action (comb-open) or
effect (plier-rip)

Children saw 8 trials in the same domain (change-of-state) and then 8 in
a new domain, directed motion. Our key interest is \textbf{\emph{not in
the learning of individual verbs}} (measured at 3), but in the biases
that children develop between verbs (measured at step 1 of each
subsequent trial). We ask (a) if children's verb biases update with
evidence within the change-of-state domain and (b) whether these biases
extend between domains, relying on an abstract means/end distinction.

We are just beginning to understand how the cognitive abilities children
show in the first year of life help to organize language learning, and
in particular how children conceptualize and break down their
representations of events into verb and sentence meaning. These results
suggest that children's verb meanings draw on very abstract lexical
semantics from childhood, and that these have parallel structure -- and
may be related to -- the fundamental cognitive representations available
to infants.}
  \keywords{keywords \\

    \indent Word count: X
  }





\usepackage{amsthm}
\newtheorem{theorem}{Theorem}
\newtheorem{lemma}{Lemma}
\theoremstyle{definition}
\newtheorem{definition}{Definition}
\newtheorem{corollary}{Corollary}
\newtheorem{proposition}{Proposition}
\theoremstyle{definition}
\newtheorem{example}{Example}
\theoremstyle{remark}
\newtheorem*{remark}{Remark}
\begin{document}

\maketitle

\setcounter{secnumdepth}{0}



Introduction outline

I. Motivate the big question/effects

Why do we have the type of linguistic system we have? Beyond question of
nature/nurture or particular syntactic theories, it's clear that
langauges make distinctions between e.g.~actions and objects.
Fundamental, built in. Why? Because they matter for cmmunication, either
how we talk about or what we want to talk about. Meets our needs.

It MATTERS which representaitonal basis we have. Effects are everywhere.
We make predictions about word meaning (Adult novel verb and
\enquote{human simulation} stuff), subtly- or not so subtly - update
meaning of words based on sentence structure (\enquote{Crash} effects
and Wolf), find verbs natural or unnatural in sentences (some rating
studies?), struggle or dont' struggle to access a word in a particular
frame (Priming). Psycholinguistic theories assume that these effects are
all driven by some shared underlying cognitive representations.
Linguistic theories provide concrete proposals for the nature of these
representations (they won't all agree that we're doing this.)

Empirical evidence for these is good for both psycholiinguistcs,
linguistics, and rest of cognition, which often struggles to describe
events and missing distinctions we believe to be important. IT PUTS A
CONSTRAINT. WE REALLY WANT EMPIRICAL SUPPORT FOR REPRESENTATIONAL FORM.
NOAH's CRYSTALLOGRAPHY METAPHOR.

Define resarch qurtion - usually talk about verb classes
(\enquote{cummunication}) but the proposal that they're broader and have
general principles and cross cutting, we explore that.

\begin{enumerate}
\def\labelenumi{\Roman{enumi}.}
\setcounter{enumi}{1}
\tightlist
\item
  Word meaning and conceptual structure.
\end{enumerate}

A. What is the content of mental representaitons of verbs? WELL Nouns.
They work like this. Conceptual components/dimensions - Take people on
the Dedre Gentner ride.

Returning to nouns, mass/count distinction implies totally independent
evidence for Spelke Objets by age 1 - they might have them much earlier,
but to the extent they really have adultlike mass count (debatable),
they have the principle.

B. How are verbs different? WELL for one thing baseball example.

Is it everything goes as far as perpsectives? Seems like NO. Use
give/receive. Or Use dimensions? CONTRAST nouns: they refer to knids,
traits tend to cluster (mutually predictive borders and hang together).
Check the cogsci version of Havasi paper 2013. Vber stend to spread,
picka dimension of carem like cause or contact.

\enquote{mental representation of events} is a bit too broad for us; as
with other cass the quyestion of whether noun representaiotn = object
representaiton is HUGE, and we leave it aside. BUT, we see evidence for
SOME kinds fo perspective taking, across langauges and theories

C. TWO PRINCIPLE OF LING THEORY TYPES vis a vis generalities. NEEDS TO
STAY SHORT!

D. WHEN GENERAL, Proposals tend to return to some common themes (cause,
agency) etc.; it's nota new idea that thesea re connected to early
cognition. WHO CARES Which is true? WELL, Theories of early cognition
also turn on questions of whether access to such representations. It
could be independent, or not, BUT THEY MUST MAP TO EACH OTHER. Thus the
linguistic achievements (if we're right about their representational
forms) of young children are a key insignt to their conceptual
structure, and the acquisition of these strucutres puts constraints on
learning.

\begin{enumerate}
\def\labelenumi{\Roman{enumi}.}
\setcounter{enumi}{2}
\tightlist
\item
  SPECIFIC REPRESENTATION TO TARGT: MANNER/RESULT
\end{enumerate}

So, how we proceed? Now for the first time in the paper talk about
Manner/Result. ANd which is it! Talmy goes here; talk about interest in
xlinguistics BUT we move on. Say explicitly that readers (lang acq) with
this background will get confused. That literature is important but not
what we're talking about.

Jesse's first paper establishes it's coherent, AND that it's learnable.
Nice, suggests we're talking about reasonable familira kinds of concepts
not some weird langauage thing. But what is the SCOPE? Put a diagram
here, probably. Why think limited? SYNTAX. NOT OBVIOUS TO LANGUAGE USER!
Why think broad? TWO INDEPENDENT STORIES, echoing the nouns again. IF WE
CAN SHOW which it is, and developmental course, can understand basis for
THIS representation, and also geenral way to understand event
representation. Cite that Brent paper that annyos me.

\begin{enumerate}
\def\labelenumi{\Roman{enumi}.}
\setcounter{enumi}{3}
\tightlist
\item
  THIS STUDIES
\end{enumerate}

We'll do 2 things. Establsih evidence for reality (replicated Havasi),
show adult. Make predictions about kids afterwards (defer to then), but
then look at developmental course. This is the roadmap.

SEPARATE FOR DISCUSION: See Behrend Farer Tomaello Gentner for this
stuff, especially on wheter we like manner or result more, which
learning ios thjere./ \enquote{Behrend wrote a second paper}. Timeline
is about 1977-185, then it goes away.

\section{Experiment 0 Experimental
Design}\label{experiment-0-experimental-design}

It's difficult! You need to understand it one time! In the epxeriments,
we'll describe deviations.

TELL PEOPLE WHAT THEY ARE CONFUSED ABOUT AND HOW NOT TO BE.

GENERAL NOTE: In analyses, make sure that item effects don't treat an
item in Causal and an item in MOtion as equivalent - there's no
pairing!!

\section{Experiment 1: Adults}\label{experiment-1-adults}

--\textgreater{}For this experiment, I'm allowed to grab text from Amy's
paper! yeyyyyy.

\subsubsection{Robust and reliable
practices}\label{robust-and-reliable-practices}

This data was previously reported as part of the second author's
dissertation.

\begin{itemize}
\tightlist
\item
  Data is available at TOADD (Need to strip MTurk IDs and birthdays if
  present)
\item
  Analysis pipeline from processing, post exclusions (based on record)
\end{itemize}

\subsection{Methods}\label{methods}

\subsubsection{Data Cleaning (to be suppressed in
submission)}\label{data-cleaning-to-be-suppressed-in-submission}

Data is loaded from cleaned scripts, post exclusion of subjects (??).
Thus, we'll need to get the info on data exclusion from the text of
Amy's dissertation\ldots{}

\subsubsection{Participants}\label{participants}

\subsubsection{Material}\label{material}

\subsubsection{Procedure}\label{procedure}

\subsubsection{Data analysis}\label{data-analysis}

We used R (3.4.1, R Core Team, 2017) for all our analyses. \#\# Results
\#\# Experiment 1 - Discussion

\section{Experiment 2: 4-5 year olds}\label{experiment-2-4-5-year-olds}

Now we do it with kids!

\subsubsection{Robust and reliable
practices}\label{robust-and-reliable-practices-1}

Way better! We report how we determined our sample size, all data
exclusions (if any), all manipulations, and all measures in the study.

\subsection{Methods}\label{methods-1}

\subsubsection{Data Cleaning (to be suppressed in
submission)}\label{data-cleaning-to-be-suppressed-in-submission-1}

Note that I need to account for the inclusion of Ss 75 and 76 (BOTH of
whom's data has to be manually entered - 1.3.17 note need to code
\emph{from video})

\begin{verbatim}
## Caught an error during read.table.
## MannerPathPriming_10.datCaught an error during read.table.
## MannerPathPriming_75.datCaught an error during read.table.
## MannerPathPriming_76.datCaught an error during read.table.
## MannerPathPriming_77.dat
\end{verbatim}

Exclusions

\begin{verbatim}
## [1] 122
\end{verbatim}

\subsubsection{Participants}\label{participants-1}

All told, the following number of participants included in each cell of
the experiment(s) are:

\begin{verbatim}
## , ,  = Action
## 
##    
##         F  M
##   3  0  1  0
##   4  0  7  6
##   5  0  6 10
##   6  0  0  1
## 
## , ,  = Effect
## 
##    
##         F  M
##   3  0  0  0
##   4  0  6  9
##   5  0  7  8
##   6  0  1  0
## 
## , ,  = Manner
## 
##    
##         F  M
##   3  0  0  0
##   4  0  7  9
##   5  0  5  6
##   6  0  1  0
## 
## , ,  = Path
## 
##    
##         F  M
##   3  0  0  0
##   4  0  6 12
##   5  0 12  2
##   6  0  0  0
## 
## , ,  = Unk
## 
##    
##         F  M
##   3  0  0  0
##   4  0  0  0
##   5  0  0  0
##   6  0  0  0
\end{verbatim}

\begin{verbatim}
## 
## Action Effect Manner   Path    Unk 
##     31     31     28     32      0
\end{verbatim}

included in the study.

\subsubsection{Materials}\label{materials}

\subsubsection{Procedure}\label{procedure-1}

\subsubsection{Data analysis}\label{data-analysis-1}

We used R (3.4.1, R Core Team, 2017) for all our analyses.

\subsection{Results}\label{results}

\subsection{Experiment 2 - Discussion}\label{experiment-2---discussion}

\section{General Discussion}\label{general-discussion}

\newpage

\section{References}\label{references}

\setlength{\parindent}{-0.5in} \setlength{\leftskip}{0.5in}

\hypertarget{refs}{}
\hypertarget{ref-R-base}{}
R Core Team. (2017). \emph{R: A language and environment for statistical
computing}. Vienna, Austria: R Foundation for Statistical Computing.
Retrieved from \url{https://www.R-project.org/}






\end{document}
